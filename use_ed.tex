\title{Why every hacker should try \lstinline{ed}}
\author{
        James Baldwin-Brown \\
}
\date{\today}

\documentclass[12pt]{article}
\usepackage{tabu}
\usepackage{listings}

\begin{document}
\maketitle

\begin{abstract}
\end{abstract}

\section{Introduction}

Much has been written about the value of the \lstinline{ed} text editor
and its importance to the history of programming culture, especially
in Unix hacker circles. The short version is that, in the late 1960s,
Ken Thompson wrote a simplified version of the \lstinline{qed} editor
to use with the nascent Unix operating system and, until Bill Joy
wrote \lstinline{vi} in 1976, it was the only widely available Unix
text editor.

\lstinline{ed} wasn't merely the only editor around, however --
it included a set of powerful features, and one in particular,
that made it one of the most useful text editors available at
that time. The extreme power and flexibility of this simple
program were instrumental in the development of the Unix philosophy,
with its emphasis on scriptability, doing one thing well, handling
text with regular expressions, and inventing little languages
to solve specific tasks. It all began with \lstinline{ed}.

\lstinline{ed} was, at least nominally, the standard Unix text editor
for several decades.
There is a recurring joke on the internet, going back to at least the early
USENET days,
that \lstinline{ed}'s place as the standard editor was a comical throwback
to a time when RAM (or \emph{core} for the elder statesmen) was measured
in kilobytes, having been superseded in terms of usability by much more
powerful editors such as \lstinline{vi} and \lstinline{emacs}.
I am not particularly interested in arguing that \lstinline{ed} is better
in some way than newer editors, but instead argue here that many of the
best qualities of the Unix philosophy are reflected by the design decisions
in its original text editor.
% \lstinline{ed} is still as usable today as
% it was in 1969, and using \lstinline{ed} for a few days or weeks will let

% \begin{table}
%     \begin{center}
%         \begin{tabu}to 0.8\linewidth {XXX}
%             Card Name & Numerical value & Alphabetic mapping\\
%             \hline\\
%             King & 0 & D\\
%             Ace & 1 & A\\
%             Deuce & 2 & K\\
%             3 & 3 & H\\
%             4 & 4 & E\\
%             5 & 5 & B\\
%             6 & 6 & L\\
%             7 & 7 & I\\
%             8 & 8 & F\\
%             9 & 9 & C\\
%             10 & 10 & M\\
%             Jack & 11 & J\\
%             Queen & 12 & G\\
%         \end{tabu}
%         \caption{Mapping to a new address space.}
%         \label{table:map}
%     \end{center}
% \end{table}

\end{document}
