\title{Why every hacker should try \lstinline{ed}}
\author{
        James Baldwin-Brown \\
}
\date{\today}

\documentclass[12pt]{article}
\usepackage{tabu}
\usepackage{listings}

\begin{document}
\maketitle

\begin{abstract}
\end{abstract}

\section{Introduction}

Much has been written about the value of the \lstinline{ed} text editor
and its importance to the history of programming culture, especially
in Unix hacker circles. The short version is that, in the late 1960s,
Ken Thompson wrote a simplified version of the \lstinline{qed} editor
to use with the nascent Unix operating system and, until Bill Joy
wrote \lstinline{vi} in 1976, it was the only widely available Unix
text editor.

\lstinline{ed} wasn't merely the only editor around, however --
it included a set of powerful features
that made it one of the most useful text editors available at
that time. The extreme power and flexibility of this simple
program were instrumental in the development of the Unix philosophy,
with its emphasis on scriptability, doing one thing well, handling
text with regular expressions, and inventing little languages
to solve specific tasks. It all began with \lstinline{ed}.

\lstinline{ed} was, at least nominally, the standard Unix text editor
for several decades.
There is a recurring joke on the internet, going back to at least the early
USENET days,
that \lstinline{ed}'s place as the standard editor was a comical throwback
to a time when RAM (or \emph{core} for the elder statesmen) was measured
in kilobytes, having been superseded in terms of usability by much more
powerful editors such as \lstinline{vi} and \lstinline{emacs}.
I am not interested in arguing that \lstinline{ed} is better
than newer editors, but instead argue here that many of the
best qualities of the Unix philosophy are reflected by the design decisions
in its original text editor.
Whatever \lstinline{ed}'s faults may be, it was the text editor used to write
the Unix kernel, shell, and userspace tools. Since the creators of Unix
wrote their own text editor, they could have written a better one if they thought one existed. It is
of note that most of the Bell labs alumni that created Unix continued
to use \lstinline{ed} long after \lstinline{vi} and other full-screen editors
became available, and eventually wrote their own replacements for \lstinline{ed} --
\lstinline{sam} and \lstinline{acme} -- that hew much closer to \lstinline{ed}'s
original regex-oriented user interface.

\section{Minimalism}

The Unix V6 \lstinline{ed} source code is only 1339 lines of assembly, which translated into 1333 lines of C in Unix V7.
Not bad for an editor that includes a full regular expression engine
and complete scriptability. This was due partially to the technical constraints
of the time (the PDP-7 on which \lstinline{ed} ran had only 4k-64k of 18-bit words in core memory),
but had more to do with a philosophy that one should use just enough code,
and no more, to complete the task at hand. This turned out to be one of the
strengths of Unix generally -- a naïve hacker could read and digest the entire
codebase because all code was as small and simple as it could be made to be. This allowed
new programmers to quickly get up to speed and begin modifying the source, even
if they didn't know the original authors.

\section{Unbroken interface}

\lstinline{ed} never pulls your attention from one place to another.
One of the big advantages that command line programmers tout about
the command line in general, and text-user-interface text editors
such as \lstinline{vi} in particular, is that they allow you to
``live'' exclusively in the terminal, never being distracted
by pop-up windows or action in another place on the screen,
and then never require you to flip through a set of windows using
\lstinline{alt + tab} just to find the thing you meant to work on.
This is true, but only to a point. When you enter \lstinline{vi},
your terminal is taken away from you in a meaningful sense. The entire
terminal view is erased and instead filled with the contents of the
text buffer you're currently editing. If you want to go back to your shell
for a minute, you can do so by a variety of methods, but that means
\lstinline{vi} has to save your terminal contents and write them
back to ther terminal so you can see them again. You can't simultaneously
see your terminal and your edited document. \lstinline{ed} never had
this problem. Because it doesn't directly manipulate your
screen, but instead just prints to \lstinline{stdout}, entering
\lstinline{ed} does not clear your screen, and any useful
information in your terminal (say, the output of `ls' or `cat')
is still available to you.

Likewise, in \lstinline{vi}, there is a command (!) that allows you
to send text to the shell to be interpreted as a shell command.
This is a cool feature because it allows you to do things like
take the contents of a line in your document, pipe it to a command,
and view the output of the command or write it back into the current buffer.
The only problem is, when you do this, you get that crazy interface jump
again -- the view switches back to your shell output, and you can't see
the document you're editing. Pressing a key will get you back to the document,
but now you can't see the shell output any more! This leads to strange
patterns in the use of \lstinline{vi}. I once saw a stackoverflow thread
in which someone asked how to see the contents of the current directory from
inside \lstinline{vim}. Rather than the simple answer (\lstinline{!ls}), the
top answer claimed that the right way to do this was to use \lstinline{:e ctrl-d}
to start to open a new buffer for editing (\lstinline{:e}), then ask \lstinline{vi}
to give you a list of autofill options for the new buffer (\lstinline{ctrl-d}).
This should not be necessary, and in \lstinline{ed}, it isn't. When you type
\lstinline{!ls} in \lstinline{ed}, you just get back 
the output of \lstinline{ls}, correctly formatted and unmoving, on the next
line of your interpreter. This allows for the \lstinline{!} command to do what
it was always intended to do: use shell programs as extensions to \lstinline{ed}
seamlessly, without any need for a built-in scripting language.
This is a common thread with \lstinline{ed}: it leans heavily on the interfaces provided
by the operating system (\lstinline{stdin}, \lstinline{stdout}, the terminal, the shell, etc.).
This isn't laziness, but rather a method for keeping the code small, the interface consistent,
and the maximum number of tools instantly available to the user. In short, it is good design.

\section{Regular expressions}

Apart from straightforward entry of text, the primary method
for editing text in \lstinline{ed} is regular expression
find-and-replace. In fact, because \lstinline{ed} is a line editor,
this is the only way to edit text without rewriting an entire line
at the minimum. First time users of \lstinline{ed} find this
constraining, but the extreme flexibility of regular expressions
allows for arbitrarily complex edits on any text. With practice,
this becomes as natural, and as quick, as any other type of interactive
editing. This type of edit is also naturally extensible to multi-line
editing, a feature that is often touted as a killer ``new'' feature of
modern text editors (see sublime text, atom, or Visual Studio Code).
\lstinline{ed} is the program that introduced the convention,
in Unix programs, that regular expressions should be surrounded
by /slashes/ and should be prefixed by commands (`s', `g') and suffixed
by options (`p', `2', `l').

\section{Scriptability}

\lstinline{ed} was intended to be used interactively. A typical \lstinline{ed}
session looks something like this:

\begin{lstlisting}
$ ls
ed_print.c
$ ed ed_print.c
93
1,$p
#include <stdio.h>

int main() {
    printf("ed is the standard editor\n");
    return(0);
}
/printf/
    printf("ed is the standard editor\n");
s/the.*ard/a good/p
    printf("ed is a good editor\n");
w
87
q
\end{lstlisting}

Here, the author opens the file ``ed\_print.c'' using ed,
prints the contents of the file (\lstinline{1,$p} indicates
that one should ``p''rint the lines from 1 until the end of the document),
searches (/) for a \lstinline{printf} line,
then ``s''ubstitutes the phrase ``a good'' for the regular
expression \lstinline{the.*ard}.
This is accomplished entirely in a unixy way -- \lstinline{ed} opens
up a file, then applies operations to the file using commands
sent to it over \lstinline{stdin}. Because it accepts commands
directly on \lstinline{stdin}, \lstinline{ed} can be used just
as easily to run pre-cooked `scripts' of \lstinline{ed} commands
as it can be used interactively. It is this killer feature that
led to its greatest influence on Unix: the creation of pipeable
shell tools and little languages.

\section{Descendants of \lstinline{ed}}

\lstinline{ed} predates the pipe, the feature of Unix that most easily allowed shell
tools to interact with one another.
Before pipes were added, \lstinline{ed} was the easiest and most
often used method for quickly automating text manipulation tasks.
When pipes were eventually added to Unix, \lstinline{ed} was the template for many
of the major pipeline tools.
\lstinline{grep} is the most famous example of this, being a neologism derived
from the common ed pattern \lstinline{g/re/p}, for ``global regular expression print''.
That's right: the most famous shell tool is a re-implementation of an \lstinline{ed}
command. Further tools, such as \lstinline{sed} and \lstinline{awk}, have similar
roots in \lstinline{ed}. From them, there is a clear throughline connecting
\lstinline{ed} all the way to modern scripting languages like \lstinline{perl} and
\lstinline{python} (\lstinline{perl} even uses \lstinline{ed}'s \lstinline{/regex/} syntax).
In many ways, \lstinline{ed} was the template for the entire Unix paradigm of integrating scripting languages
with compiled code for rapid prototyping, portability, and speed.

\lstinline{ed} didn't just influence scripting languages. Through the lexer and parser
programs \lstinline{lex} and \lstinline{yacc} it directly influenced the way that
languages are implemented. \lstinline{lex} is a lexing program that feeds
language tokens to \lstinline{yacc}, a language parser. \lstinline{lex} code consists
primarily of a set of \lstinline{ed}-style regular expressions that specify
the types of tokens that the language should accept. These programs were used to implement
some of the above languages, most notably \lstinline{awk}. So, \lstinline{awk} is an ed-influenced
regular-expression-driven reporting language, written in an ed-influenced, regular-expression-driven lexer/parser toolset,
and the code for all of these tools was written in \lstinline{ed}. It's \lstinline{ed} all the way down!

\section{Some takeaways}

If you take one thing from this essay, it should not just be that \lstinline{ed} is
a masterpiece of minimalist design, or that it is one of the most influential programs
ever written, though it is both. What you should take away is that using \lstinline{ed}
is a way to use Unix as intended by the people that designed
the entire system. Don't worry about using it all the time -- I certainly don't. But try
using it for a few days, even a few weeks. It may be frustrating at first, but using
it will show you just how powerful a few carefully chosen lines of C code can be.
After using it for a while, you will see all text as beautifully nested sets of
regular expressions, ready to be taken apart and put back together with a single
\lstinline{s/re/re/g}. You'll scoff at the so-called ``power users'' of other editors,
with their clunky point-and-click interfaces and unneccessary extensions.
If you don't believe me about any of this, at least believe that \lstinline{ed} really
is a useful editor -- I used it to write this entire document.



% \lstinline{ed} is still as usable today as
% it was in 1969, and using \lstinline{ed} for a few days or weeks will let

% \begin{table}
%     \begin{center}
%         \begin{tabu}to 0.8\linewidth {XXX}
%             Card Name & Numerical value & Alphabetic mapping\\
%             \hline\\
%             King & 0 & D\\
%             Ace & 1 & A\\
%             Deuce & 2 & K\\
%             3 & 3 & H\\
%             4 & 4 & E\\
%             5 & 5 & B\\
%             6 & 6 & L\\
%             7 & 7 & I\\
%             8 & 8 & F\\
%             9 & 9 & C\\
%             10 & 10 & M\\
%             Jack & 11 & J\\
%             Queen & 12 & G\\
%         \end{tabu}
%         \caption{Mapping to a new address space.}
%         \label{table:map}
%     \end{center}
% \end{table}

\end{document}
